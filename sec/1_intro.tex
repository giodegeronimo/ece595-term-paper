% 1. Introduction
% Briefly define the broader theme connecting the papers: efficient representations and transformation of visual data.
% Motivate why generative, segmentation, and temporal models illustrate different tradeoffs.
% State contributions of your term paper:
% Structured critique of four recent CV/ML models.
% Nontrivial implementation and evaluation of Rectified Flow sampling.


% 1. Introduction
% Purpose: Set the stage for why these papers belong in the same story, and what contribution your term paper makes.
% What goes in it:
% Context: One short paragraph stating a shared high-level theme among your chosen papers. For you, that theme is learning efficient visual representations and transformations — but across different problem settings (generative modeling, video action segmentation, flexible ViTs, and universal segmentation).
% Motivation: Explain why this general problem matters — e.g., increasing scale of visual data, domain constraints, compute costs.
% Your task: State that you read and analyzed four recent papers and implemented one (Rectified Flow).
% Your contributions: A clear 3-item bullet list (still in prose):
% Comparative critique of the four methods.
% Implementation of Rectified Flow.
% Empirical evaluation and discussion.
% Length: ~3–5 paragraphs total.
% Absolutely do NOT preview all details here. Just set context + contributions.

\section{Introduction}
\label{sec:intro}

1. Introduction
Briefly define the broader theme connecting the papers: efficient representations and transformation of visual data.
Motivate why generative, segmentation, and temporal models illustrate different tradeoffs.
State contributions of your term paper:
Structured critique of four recent CV/ML models.
Nontrivial implementation and evaluation of Rectified Flow sampling.