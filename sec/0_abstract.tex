% 4–6 sentences.
% One sentence each on:
% Problem domain.
% What the selected papers address.
% Key insight of your critique.
% What you implemented + dataset + evaluation result.
% One-line takeaway.

\begin{abstract}
The field of computer vision is a fast paced environment with twists and turns at every direction. In this paper I will cover the content of three interesting papers across different fields of computer vision. \cite{cheng2022mask2former} introduces a unified architecture for semantic, instance, and panoptic segmentation problems that outperforms specialized models. \cite{lu2024fact} introduces a novel method for action segmentation in videos that involves two branches that communicate with each other to refine their predictions. \cite{dehghani2023navit} introduces a method of training vision transformers to train on variable resolution/aspect ratio images. Finally, \cite{liu2023rectifiedflow} introduces a method for learning ODEs for domain transfer that follow straight lines, offering reduced inference costs in tasks such as image generation. In this paper I will offer my perspective on the strengths, weaknesses, assumptions, novelty, clarity, and experimental rigor of each of these papers. I will then do XYZ as a non-trivial implementation of paper ABC.
\end{abstract}