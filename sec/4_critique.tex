\section{Critical Comparative Analysis}
\label{sec:critique}

% No paragraph before subsections. Each subsection gets multiple paragraphs.

%-------------------------------------------------------------------------
\subsection{Problem Motivation and Scope}

Assess how clearly each paper justifies its target problem.
Compare generality vs. specificity of problem framing.
Evaluate which motivations are strongest and why.

%-------------------------------------------------------------------------
\subsection{Novelty and Conceptual Contributions}

Determine what is genuinely new vs. incremental.
Analyze conceptual depth vs. architectural tuning.
Position the contributions relative to prior work.

%-------------------------------------------------------------------------
\subsection{Methodological Design and Assumptions}

Discuss design choices and trade-offs (e.g., complexity, stability).
Identify implicit assumptions that limit applicability.
Compare model elegance and internal consistency across papers.

%-------------------------------------------------------------------------
\subsection{Experimental Rigor and Evaluation Validity}

Evaluate dataset selection, baselines, metrics, and ablations.
Check whether claimed improvements are statistically meaningful.
Assess fairness and reproducibility of evaluations.

%-------------------------------------------------------------------------
\subsection{Limitations and Generalization}

Identify failure modes, scalability issues, or domain constraints.
Discuss where the methods likely underperform in practical use.

%-------------------------------------------------------------------------
\subsection{Synthesis and Comparative Positioning}

Compare the papers across conceptual and practical axes.
Discuss which contributions are likely to have lasting influence.
Provide a concise ranking or positioning argument grounded in evidence.
