\section{Critical Comparative Analysis}
\label{sec:critique}

% No paragraph before subsections. Each subsection gets multiple paragraphs.

%-------------------------------------------------------------------------
\subsection{Problem Motivation and Scope}

Assess how clearly each paper justifies its target problem.
Compare generality vs. specificity of problem framing.
Evaluate which motivations are strongest and why.

%-------------------------------------------------------------------------
\subsection{Novelty and Conceptual Contributions}

Determine what is genuinely new vs. incremental.
Analyze conceptual depth vs. architectural tuning.
Position the contributions relative to prior work.

%-------------------------------------------------------------------------
\subsection{Methodological Design and Assumptions}

% Discuss design choices and trade-offs (e.g., complexity, stability).
% Identify implicit assumptions that limit applicability.
% Compare model elegance and internal consistency across papers.
NaViT's \cite{dehghani2023navit} design choices naturally have tradeoffs. While their sequence packing training method greatly reduces padding waste, it introduces overhead in the data loader, as it must greedily pack the images into multiple sequences and pad them to a fixed length. Variable token dropping significantly reduces training costs, but risks losing important image details if done too aggressively. Some assumptions made by NaViT can also limit its applicability. The claim that attention can learn scale/geometry without fixed grids assumes that the model and data are sufficient to learn scale/translation behavior that CNNs get ``for free.'' With small datasets, performance may lag behind standard CNNs. The authors hint at this in their Limitations section, noting that experiments were only performed in the large-dataset regime. They also assume that token dropping won't remove critical information, which only holds when tasks are resilient to sparse spatial sampling. Overall, though, NaViT is conceptually sound. They use the same ubiquitous transformer stack but with some extra training practices. Added complexity sits mostly in input/packing logic, not in the backbone itself.  

%-------------------------------------------------------------------------
\subsection{Experimental Rigor and Evaluation Validity}

Evaluate dataset selection, baselines, metrics, and ablations.
Check whether claimed improvements are statistically meaningful.
Assess fairness and reproducibility of evaluations.

%-------------------------------------------------------------------------
\subsection{Limitations and Generalization}

Identify failure modes, scalability issues, or domain constraints.
Discuss where the methods likely underperform in practical use.

%-------------------------------------------------------------------------
\subsection{Synthesis and Comparative Positioning}

Compare the papers across conceptual and practical axes.
Discuss which contributions are likely to have lasting influence.
Provide a concise ranking or positioning argument grounded in evidence.
